\documentclass{paper}

\usepackage[margin=1in]{geometry}

\title{Perspectives on Reproducibility and Sustainability of Open Source Scientific Software from Six Years of the Dedalus Project}
\author{The Dedalus Collective}
\begin{document}
\maketitle

\section{Introduction}
\label{sec:intro}

Background on Dedalus; mission, code, users.

\section{Reproducibility}
\label{sec:repro}

Verification and validation: who can do it? With an open code, anyone can perform V\&V exercises. 

\section{Maintenence and User Support}
\label{sec:support}

What are the additional burdens placed on developers because of open source? A difficult question to answer directly, but can come at this obliquely by thinking about the \emph{total amount of time} we spend answering user questions. How does this burden the developers? What \emph{rewards} if any does doing this kind of support do? Do we understand the code better because of it (think of wave-driving example)? 

\section{Software stack}
\label{sec:stack}

The use of a completely open source stack/toolchain (we need to avoid too many buzzwords here) significantly affects the workflow, for both good and bad. On the good side, we are encouraged to tap into a very large infrastructure of open source things (jupyter, docker, possible cocalc integration) that can blend research at many scales (simple 1D things running in the cloud/on a laptop all the way to supercomputing) and education. On the bad side, installation is a nightmare; something that open source communities often do not focus on enough.

\appendix

\section{Supporters}
\label{sec:supporters}

``A numbered list of supporters who did not contribute significantly to the text'' goes here.

\end{document}
